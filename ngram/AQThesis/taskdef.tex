
\section{ Постановка задачи }\label{sec:taskdef}

\begin{definition}{\textit{Оптическое распознавание символов (Optical Character Recognition, OCR)}} -- процесс считывания текста с физического носителя и его сохранения в цифровом формате. Текст состоит из \textit{символов}.
\end{definition}

\begin{definition}{\textit{Ошибка OCR}} -- случай, когда очередной символ текста распознался неверно или не распознался. Ведёт к понижению качества распознавания.
\end{definition}

\begin{definition}{\textit{$N$-грамма}} -- последовательность из $n$ элементов (слов, звуков, символов). Анализируя их частотности, можно строить модели для анализа и синтеза языка.
\end{definition}

\begin{definition}{\textit{$N$-граммная модель}} -- вероятностная модель языка, которая рассчитывает вероятность последнего элемента $n$-граммы, если известны все предыдущие. \\
При использовании $n$-граммных моделей предполагается, что появление каждого элемента зависит только от предыдущих элементов.
\end{definition}

\textbf{Цель работы} -- сравнить эффективность различных символьных $n$-граммных моделей в задаче исправления ошибок OCR в японском языке.

Из цели работы вытекают следующие \textbf{задачи}:
\begin{itemize}
	\item Рассмотреть существующие подходы к $n$-граммному моделированию японского языка;
	
	\item Реализовать некоторые модели;
	
	\item Развернуть систему для тестирования и сравнения моделей.
\end{itemize}

Чтобы понять специфику цели работы, нужно учесть особенности японского языка.

Очевидно, что устройство японского языка на уровне конкретных символов сложнее, чем устройство языков латино-романской группы, в которых существует всего 25-40 символов, учитывая возможную диакритику.

\subsection{ Обзор японского языка }
\label{sec:japanese}

Письменный японский текст -- это комбинация слогово-фонетических символов (кана) и иероглифов (кандзи).
Слоговая азбука кана делится на катакану и хирагану, которые представляют собой разные графические формы одних и тех же слогов.

Рассмотрим эти символы подробнее:

\begin{itemize}
	\item Хирагана (см. Рис. \cref{fig:hirag_sample}). В основном используется для образования грамматических морфем.
	\begin{figure}[H]
		\centering
		\includegraphics{hirag_sample.png}
		\caption{hirag\_sample}
		\label{fig:hirag_sample}
	\end{figure}
	
	\item Катакана (см. Рис. \cref{fig:katak_sample}). Используется для транскрибирования иностранных заимствованных слов.
	\begin{figure}[H]
		\centering
		\includegraphics{katak_sample.png}
		\caption{katak\_sample}
		\label{fig:katak_sample}
	\end{figure}

	\item Также есть диакритические символы -- дакутен, хандакутен (см. Рис. \cref{fig:dakut_sample_hir} и \cref{fig:dakut_sample_kat}). Они могут применяться как к катакане, так и к хирагане, и определённым образом влияют на звучание слогов.
	\begin{multicols}{2}
		
		\begin{figure}[H]
			\centering
			\includegraphics[scale=0.6]{dakut_sample.png}
			\caption{dakut\_sample}
			\label{fig:dakut_sample_hir}
		\end{figure}
		
		\begin{figure}[H]
			\centering
			\includegraphics[scale=0.6]{handakut_sample.png}
			\caption{dakut\_sample}
			\label{fig:dakut_sample_kat}
		\end{figure}
		
	\end{multicols}
	
	\item Кандзи (см. Рис. \cref{fig:kandji_sample}). Это символы, несущие семантическую нагрузку.
	\begin{figure}[H]
		\centering
		\includegraphics{kanji_sample.png}
		\caption{kandji\_sample}
		\label{fig:kandji_sample}
	\end{figure}
\end{itemize}

Кана различает 46 слогов, которые могут записываться как катаканой, так и хираганой. А вот иероглифов кандзи существует гораздо больше (2136 (т.н. jōyō kanji -- "обычно используемые кандзи") достаточно для жизни, 6879 используется в кодировке JIS X 0208 (Japanese Industrial Standart, см. \cite{JISX0208}), а стандарт Unicode определяет то ли 21000, то ли 75000, надо разобраться) \todo{цифры, мб зависит от стандарта Unicode?}.

Кроме перечисленных символов, в японском тексте могут быть и другие: фуригана -- маленькие знаки каны в качество фонетических подсказок, ромадзи -- система транслитерации японских слов в латиницу и т.д. Однако, в данной работе эти разделы оставлены за кадром.

Японский текст записывается с помощью комбинаций кандзи, кан и пунктуации, при этом отсутствует пробельное деление предложений на слова (см. Рис. \cref{fig:japtext_sample}).
	\begin{figure}[H]
	\centering
	\includegraphics{japtext_sample.png}
	\caption{japtext}
	\label{fig:japtext_sample}
\end{figure}

По сравнению с латино-романскими языками, где алфавит меньше в сотни раз, а деление текста на слова очевидно, задача корректного распознавания символов становится значительно сложнее. Это требует более изощрённых подходов для автоматического анализа распознанного текста и поиска ошибок в нём.

Рассмотрим несколько примеров символов, которые легко спутать.

\subsection{ Путающиеся символы в японском }

При таком большом размере алфавита частые ошибки OCR можно разбить по классам. Вот некоторые из них:

\begin{itemize}
	\item[\textbf{2Kana}] 2 похожие каны. Таких случаев достаточно мало, а методы их различения уже существуют (см., например, метод с использованием глубокого обучения и свёрточных нейронных сетей в работе \cite{tsai:dcnn}).
	\begin{figure}[H]
		\center{\raisebox{-.5\height}{\includegraphics[height=100pt]{KanaO.png}}\ и \raisebox{-.5\height}{\includegraphics[height=100pt]{KanaWi.png}}}
	\end{figure}

	\item[\KG] Кана может легко путаться с соответствующим её дакутен-символом.
	\begin{figure}[H]
		\center{\raisebox{-.5\height}{\includegraphics[height=100pt]{KanaKa.png}}\ и \raisebox{-.5\height}{\includegraphics[height=100pt]{KanaGa.png}}}
	\end{figure}

	\item[\BS] Существуют большие и маленькие каны, которые нужно различать.
	\begin{figure}[H]
		\center{\raisebox{-.5\height}{\includegraphics[height=100pt]{KanaYo.png}}\ и \raisebox{-.5\height}{\includegraphics[height=100pt]{KanaYoSmall.png}}}
	\end{figure}
	
\end{itemize}

В будущем мы будем рассматривать 3 случая ошибок: \textbf{\KG}, \textbf{\BS} и \textbf{\MX} (смесь \textbf{\KG} и  \textbf{\BS} ), более строгое определение которых будет дано в \cref{sec:experiment}.

\subsection{ Формальная постановка задачи }

\begin{definition}
	{\textit{Алфавит $\Sigma = \{ a, b, c, .. \}$}} -- множество символов в данном языке. В японском языке их около 80000, стандарт Unicode поддерживает примерно 21000.
\end{definition}

\begin{definition}
	{\textit{Текст $Text \in \Sigma^+$}} -- последовательность символов из алфавита $\Sigma$ положительной длины.
\end{definition}

\begin{definition}
	Текст делится на конечное множество {\textit{предложений $S = \{ S_1, S_2, S_3, ... \}$}} знаками пунктуации и форматированием. $Text = S_1S_2S_3...$.
\end{definition}

Для каждого из предложения текста существует единственно верный вариант написания, а также некоторое (фиксированное) число неверных. Требуется ответить, какой из вариантов верен.

\begin{definition}
	{\textit{Оценивающий алгоритм (estimator) $\Theta : S \rightarrow \mathbb{R}^+ $}} -- функция, возвращающая оценку правильности варианта $S$.
\end{definition}

Среди $k$ вариантов предложения выбирается наилучший: $S_{best} = \argmaxl_{S} \Theta(S)$, который и считается правильным.

Если $S_{best}$ угадано верно, то на данном предложении алгоритм $\Theta$ отработал правильно.

\begin{definition}
	{\textit{Качество алгоритма $Q(\Theta) = \dfrac{\#\{ \text{угаданных предложений} \}}{\#\{ \text{всего предложений} \}}$}}.
\end{definition}

Задача -- реализовать ряд оценивающих алгоритмов (см. раздел \cref{sec:models}), основанных на $n$-граммных моделях, и сравнить их по качеству.