\section{ Результаты эксперимента }\label{sec:results}

Напомним условия эксперимента. Модели Ngram($n = 1$), Backoff($n = 5$) и Katz($n = 5$) запускались на тестовых выборках размером $\approx 300 MB$, что соответствует $\approx 8 \cdot 10^6$ предложений, или же $\approx \dfrac{1}{10}$ части корпуса. При этом размер алфавита был равен $|\Sigma{B^*}| = 4800$, выборки были зашумлены 3 различными способами: \KG, \BS\ и \MX. 

\subsection{ Результаты для различных моделей }

Первая часть эксперимента проводилась без оценивания уверенности ответов (см. \cref{sec:confidencemodel}) и показала следующие результаты:

\vspace{20pt}

\begin{tabular}{|c|c|c|c|}\hline
	$M \backslash N$ & \KG & \BS & \MX \\ \hline
	Ngram(1) (Baseline) & 0.75 & 0.88 & 0.76 \\
	Backoff(5) & 0.89 & 0.93 & 0.90 \\
	Katz(5) & 0.961 & 0.965 & 0.962 \\ \hline 	
\end{tabular}

\vspace{20pt}

\todo{сюда ли красивую таблицу со статистиками по канам? и надо ли её вообще? мб просто пару примеров и описать словами?}

После того, как в первой части эксперимента хорошие результаты показала модель Katz(5), для неё были проведены испытания по поиску оптимального уровня уверенности $Confidence, C$:

\vspace{20pt}

\begin{multicols}{2}
	\begin{tabular}{|c|c|c|c|}\hline
		$C \backslash N$ & \KG & \BS & \MX \\ \hline
		0.9 & 0.86 & 0.94 & 0.86 \\
		0.95 & 0.979 & 0.988 & 0.981 \\
		0.97 & 0.966  & 0.986 & 0.967  \\
		0.99 & 0.94 & 0.98 & 0.95 \\ \hline 	
	\end{tabular}

	\begin{tabular}{|c|c|c|c|}\hline
		$C \backslash N$ & \KG & \BS & \MX \\ \hline
		0.9 & 0.53 & 0.43 & 0.52 \\
		0.95 & 0.61 & 0.69 & 0.62 \\
		0.97 & 0.77 & 0.85  & 0.77  \\
		0.99 & 0.91 & 0.94 & 0.91 \\ \hline 	
	\end{tabular}
\end{multicols}

\vspace{20pt}

В итоге оптимальная конфигурация -- модель Катца с $n=5$ и уверенностью $С=0.97$ ( Katz($n=5, C=0.97$) ). Она даёт высокие результаты исправления ошибок ($0.97-0.99$), классифицируя достаточно большую часть выборки ($77-85 \%$). Результаты работы этой конфигурации в будут более детально проанализированы в \cref{sec:analysis}.

\subsection{ Затраты ресурсов }

Ресурсы делятся на время и память, затраты которых определяются, в нашем случае, глубиной бора $n$ и размером тестовой выборки $size$.

Ниже представлены сводные таблички для различных моделей.

\todo{таблички, мб теор. асимптотики}