%%%%%%%%%%%%%%%%%%%%%%%%%%%%%%%%%%%%%%%%%%%%%%%%%%%%%%%%%%%%%%%%%%%%%%
% LaTeX Template: Curriculum Vitae
%
% Source: http://www.howtotex.com/
% Feel free to distribute this template, but please keep the
% referal to HowToTeX.com.
% Date: July 2011
% 
%%%%%%%%%%%%%%%%%%%%%%%%%%%%%%%%%%%%%%%%%%%%%%%%%%%%%%%%%%%%%%%%%%%%%%
% How to use writeLaTeX: 
%
% You edit the source code here on the left, and the preview on the
% right shows you the result within a few seconds.
%
% Bookmark this page and share the URL with your co-authors. They can
% edit at the same time!
%
% You can upload figures, bibliographies, custom classes and
% styles using the files menu.
%
% If you're new to LaTeX, the wikibook is a great place to start:
% http://en.wikibooks.org/wiki/LaTeX
%
%%%%%%%%%%%%%%%%%%%%%%%%%%%%%%%%%%%%%%%%%%%%%%%%%%%%%%%%%%%%%%%%%%%%%%
\documentclass[paper=a4,fontsize=11pt]{scrartcl} % KOMA-article class

\usepackage[english]{babel}
\usepackage[utf8x]{inputenc}
\usepackage[protrusion=true,expansion=true]{microtype}
\usepackage{amsmath,amsfonts,amsthm}     % Math packages
\usepackage{graphicx}                    % Enable pdflatex
\usepackage[svgnames]{xcolor}            % Colors by their 'svgnames'
\usepackage{geometry}
\textheight=700px                    % Saving trees ;-)
\usepackage{url}

\frenchspacing              % Better looking spacings after periods
\pagestyle{empty}           % No pagenumbers/headers/footers

\usepackage[colorlinks = true,
linkcolor = blue,
urlcolor  = blue,
citecolor = blue,
anchorcolor = blue]{hyperref}

%%% Custom sectioning (sectsty package)
%%% ------------------------------------------------------------
\usepackage{sectsty}

\sectionfont{%			            % Change font of \section command
	\usefont{OT1}{phv}{b}{n}%		% bch-b-n: CharterBT-Bold font
	\sectionrule{0pt}{0pt}{-5pt}{3pt}}

%%% Macros
%%% ------------------------------------------------------------
\newlength{\spacebox}
\settowidth{\spacebox}{888888888888888888888888}			% Box to align text
\newcommand{\sepspace}{\vspace*{1em}}		% Vertical space macro

\newcommand{\MyName}[1]{ % Name
	\Huge \usefont{OT1}{phv}{b}{n} \hfill #1
	\par \normalsize \normalfont}

\newcommand{\MySlogan}[1]{ % Slogan (optional)
	\large \usefont{OT1}{phv}{m}{n}\hfill \textit{#1}
	\par \normalsize \normalfont}

\newcommand{\MyProject}[1]{ % Slogan (optional)
	\Large \usefont{OT1}{phv}{m}{n}\hfill #1
	\par \normalsize \normalfont}

\newcommand{\MyMentor}[1]{ % Slogan (optional)
	\large \usefont{OT1}{phv}{m}{n}\hfill \textit{Mentor: #1}
	\par \normalsize \normalfont}

\newcommand{\NewPart}[1]{\section*{\uppercase{#1}}}

\newcommand{\PersonalEntry}[2]{
	\noindent\hangindent=2em\hangafter=0 % Indentation
	\parbox{\spacebox}{        % Box to align text
		\textit{#1}}		       % Entry name (birth, address, etc.)
	\hspace{1.5em} #2 \par}    % Entry value

\newcommand{\SkillsEntry}[2]{      % Same as \PersonalEntry
	\noindent\hangindent=2em\hangafter=0 % Indentation
	\parbox{\spacebox}{        % Box to align text
		\textit{#1}}			   % Entry name (birth, address, etc.)
	\hspace{1.5em} #2 \par}    % Entry value	

\newcommand{\EducationEntry}[4]{
	\noindent \textbf{#1} \hfill      % Study
	\colorbox{Black}{%
		\parbox{6em}{%
			\hfill\color{White}#2}} \par  % Duration
	\noindent \textit{#3} \par        % School
	\noindent\hangindent=2em\hangafter=0 \small #4 % Description
	\normalsize \par}

\newcommand{\WorkEntry}[4]{				  % Same as \EducationEntry
	\noindent \textbf{#1} \hfill      % Jobname
	\colorbox{Black}{\color{White}#2} \par  % Duration
	\noindent \textit{#3} \par              % Company
	\noindent\hangindent=2em\hangafter=0 \small #4 % Description
	\normalsize \par}

%%% Begin Document
%%% ------------------------------------------------------------
\begin{document}
	% you can upload a photo and include it here...
	%\begin{wrapfigure}{l}{0.5\textwidth}
	%	\vspace*{-2em}
	%		\includegraphics[width=0.15\textwidth]{photo}
	%\end{wrapfigure}
	
	\MyName{Alexey Kulikov}
	\MySlogan{GSoC Project Proposal}
	\MyProject{Regression localization from stack traces}
	\MyMentor{Mr. Calixte Denizet}
	
	\sepspace
	
	%%% Personal details
	%%% ------------------------------------------------------------
	\NewPart{Personal Information}{}
	
	\PersonalEntry{Name}{Alexey Kulikov}
	\PersonalEntry{Email}{\url{alexeyqu@gmail.com}}
	\PersonalEntry{IRC nick}{alexeyqu}
	\PersonalEntry{Telephone}{+7 (929) 639-99-14 }
	\PersonalEntry{Website}{\href{https://alexeyqu.github.io/}{alexeyqu.github.io} }
	\PersonalEntry{Other contact methods}{\href{http://github.com/alexeyqu}{github}}
	\PersonalEntry{Country of residence}{Russia}
	\PersonalEntry{Timezone}{UTC+3}
	\PersonalEntry{Primary language}{Russian}
	
	%%% Education
	%%% ------------------------------------------------------------
	\NewPart{Project Proposal}{}
	
	\begin{itemize}
		\item The goal is to develop a tool that can quickly find a patch that caused a bug.
		
		\item This tools will be analyzing a set of stack traces, performing the intersection with the Firefox call graph and a set of recently committed patches. 
		
		\item One of the best ways to do it --- a Clang plugin, due to its vast and customizable API providing ways to work with AST, parse table and llvm libs. 
		
	\end{itemize}
	

	\NewPart{Schedule of Deliverables}
	
	The final deliverables would consist of: 
	
	\begin{itemize}
		\item CLang plugin providing high quality visualizations of call graphs; 
		\item Tools providing interaction with Bugzilla;
		\item Tools for mapping bugs on specific patches in Bugzilla;
		\item Some test cases.
	\end{itemize}
	
	Staring on May 4, when Google announces the accepted students, the whole available time of GSoC is about 16 weeks. Accordingly, my planned schedule is:
	
	\begin{itemize}
		\item  Weeks 1-3: Getting to know the community; discussing the details with my mentor; 
		\item  Weeks 3-5: Studying the Clang API better, starting to develop Backtrace analyzer;
		\item  Weeks 5-7: Connecting the tool to Bugzilla, getting some test cases;
		
		Note: Here I've got my Bachelor Thesis defence, it will take some of my time;
		
		\item  Weeks 7-11: Setting up the interaction with latest patches;
		\item  Weeks 11-14: Testing with different cases, updating the code accordingly;
		\item  Weeks 14-16: Tying up loose ends, writing docs, pushing onto public repos.
	\end{itemize}
	
	\NewPart{Open Source Development Experience}
	
	I've participated in several student projects hosted on github \\(e.g. \href{https://github.com/team-PowerSpace/PowerSpace}{Powerspace -- Powerpoint for mindmaps.}) \\
	as well as developed several on my own \\(e.g. implementation of the classic \href{https://github.com/alexeyqu/OntoTeamCompiler}{MiniJava Compiler}).
	
	\NewPart{Work/Internship Experience}
	
	I am presently on leave from my normal job as a computational linguist at ABBYY, \\so that I can complete my Bachelor Degree. 
	
	\sepspace
	
	\EducationEntry{Computational Linguist -- ABBYY}{2016-2017}{
	\begin{itemize}
		\item Developing information extraction rules for ABBYY Compreno technology;
		\item Resolved over 100 issues on extraction of basic concepts;
		\item Successfully deployed 2 proof-of-concept projects for corporate clients.
	\end{itemize}}

	\EducationEntry{Teaching Assistant -- «Intellectual» school}{2013-2016}{
		\begin{itemize}
		\item One of top 10 schools in Russia in 2016;
		\item Led Formal Languages and Math Olympiad courses;
		\item 7 students got several awards in a few months.
	\end{itemize}}
	
	%%% Work experience
	%%% ------------------------------------------------------------
	\NewPart{Academic Experience}{}
	
	\EducationEntry{Student -- MIPT}{2013-now}{
		\begin{itemize}
			\item Majoring in CS;
			\item Writing bachelor thesis on kana/kanji OCR error correction;
			\item Participating in different study projects and activities\\ (e.g. \href{http://julesvernetrilogy.com/}{Jules Verne Trilogy Visualization});
			\item Running, swimming, playing the guitar.
		\end{itemize}
	}
	
	%%% Skills
	%%% ------------------------------------------------------------
	\NewPart{Why Me}{}
	
	\sepspace
	
	I have some experience in programming (CPU emulation + binary translation to x86, experiments with OpenGL, the (already mentioned) MiniJava compiler, game bots development and some more)
	as well as in NLP (during my work at ABBYY) and machine learning.
	
	\sepspace
	
	I believe that Clang is a really good tool for understanding the internals of a compiler.
	
	I want to learn it more.


\newpage

	\sepspace
	\sepspace
	
	Some of my relevant skills are listed below:
	
	\sepspace
	
	\SkillsEntry{Languages}{Russian (native)}
	\SkillsEntry{}{English (fluent)}
	\SkillsEntry{}{French (intermediate, DELF B1 diploma,}
	\SkillsEntry{}{participant of All-Russian Olympiad Finals)}
	
	\SkillsEntry{Programming}{C/C++ (flex-bison, Win32 API, Doxygen)}
	\SkillsEntry{}{Python (sklearn, numpy, scipy, nltk)}
	\SkillsEntry{}{Unix (bash, System V IPC)}
	\SkillsEntry{}{\LaTeX}
	
	\sepspace
	
	If you want to know more, please look at my CV (available \href{http://alexeyqu.github.io/}{here}).
	
	\NewPart{Why Mozilla}
	
	\sepspace
	
	I really love the spirit of Mozilla and would like to be part of its community. \\This project is a good start.
	
	\sepspace
	
	Plus, I've already applied to Mozilla Summer Internship for quite a similar topic, because it is really interesting and useful.
	
	
	%%% References
	
	
	
\end{document}
