\documentclass[14pt]{beamer}
\usepackage[T2A]{fontenc}
\usepackage[utf8]{inputenc}
\usepackage[english,russian]{babel}
\usepackage{amssymb,amsfonts,amsmath,mathtext}
\usepackage{cite,enumerate,float,indentfirst}
\usepackage{booktabs}
\usepackage{listings}

\graphicspath{{images/}}

\usetheme{Pittsburgh}
\usecolortheme{whale}

\setbeamercolor{footline}{fg=blue}
\setbeamertemplate{footline}{
	\leavevmode%
	\hbox{%
		\begin{beamercolorbox}[wd=.333333\paperwidth,ht=2.25ex,dp=1ex,center]{}%
			Kulikov Alexey, ABBYY
		\end{beamercolorbox}%
		\begin{beamercolorbox}[wd=.333333\paperwidth,ht=2.25ex,dp=1ex,center]{}%
			Moscow, 2017
		\end{beamercolorbox}%
		\begin{beamercolorbox}[wd=.333333\paperwidth,ht=2.25ex,dp=1ex,right]{}%
			Page \insertframenumber{} of \inserttotalframenumber \hspace*{2ex}
	\end{beamercolorbox}}%
	\vskip0pt%
}

\newcommand{\itemi}{\item[\checkmark]}

\title{Flex}
\subtitle{\footnotesize{}}
\author{\small{%
		~Kulikov Аlexey}\\%
	\vspace{30pt}%
	ABBYY%
	\vspace{20pt}%
}
\date{\small{Moscow, 2017}}

\begin{document}
	
	\maketitle
	
	\begin{frame}
		
		\frametitle{What is Flex?}
		\begin{itemize}
			\item Flex -- The Fast Lexical Analyzer;
			\item A tool for generating programs that perform pattern-matching on text;
			\item Flex is a free implementation of the original Unix \textbf{lex} program;
			\item Free, but non-GNU.
			\item \href{https://github.com/westes/flex}{github.com/westes/flex}
		\end{itemize}
	\end{frame}

	\begin{frame}
	
		\frametitle{What is lex?}
		\begin{itemize}
			\item Lex -- A Lexical Analyzer Generator;
			\item A tool for writing programs that are directed \\by regex instances from the input;
			\item Is designed to simplify interfacing with Yacc;
			\item The <<host language>> can be \textbf{C}, Ratfor, ...
			\item \href{http://dinosaur.compilertools.net/}{dinosaur.compilertools.net}
		\end{itemize}
	\end{frame}
		
	\begin{frame}
	
		\frametitle{Lex overview}
		\begin{itemize}
			\item Lex source -- a table ( regex $\rightarrow$ action );
			\item Lex generates a DFA based on the table;
			\item DFA parses the input line by line \\and splits lines into tokens;
			\item If a token matches some \textbf{regex}, \\the corresponding \textbf{action} occurs.
		\end{itemize}
	\end{frame}	

	\begin{frame}
	
		\frametitle{Lex overview}
		\begin{itemize}
			\item Lex accepts ambiguous specifications and chooses \textbf{the longest match} possible;
			\item All the actions occur when a line is fully parsed;
			\item The order of actions is preserved;
			\item If no match found, the token is printed <<as is>>.
		\end{itemize}
	\end{frame}

	\begin{frame}
		
		\frametitle{Lex pipeline}
		\begin{itemize}
			\item[1] $ \text{Source} \rightarrow \boxed{\text{ Lex }} \rightarrow \text{yylex} $
			\item[2] $ \text{Input} \rightarrow \boxed{\text{ yylex }} \rightarrow \text{Output} $
		\end{itemize}
	\end{frame}

\begin{frame}[fragile]
	
	\frametitle{Lex source code}	
	
	\begin{lstlisting}
  {definitions}
  %%
  {rules}
  %% // (optional)
  {user subroutines} // (optional)
	\end{lstlisting}
	
\end{frame}

\begin{frame}[fragile]

\frametitle{A trivial example}
	
\begin{itemize}
	\item Strips whitespace at the EOL;
	\item Squashes all other whitespace characters.
\end{itemize}	

	\begin{lstlisting}
	%%
	[ \t]+$   ; 
	[ \t]+    printf(" ");
	\end{lstlisting}

	
\end{frame}

\begin{frame}[fragile]
	
	\frametitle{Regex}	
	
	Operators: \text{" \ [ ] \textasciicircum - ? . * + | ( ) \$ / \{ \} \% < >}
	
	\begin{itemize}
		\item $xyz"++" \Leftrightarrow xyz \backslash+\backslash+$
		\item $[a-z0-9<>\_]$
		\item $[\backslash40-\backslash176]$ -- printable ASCII
		\item $(ab|cd+)?(ef)*$
		\item $ab/cd$ -- trailing context
		\item $\hat{}\ abc\$ \Leftrightarrow \hat{}\ abc/\backslash n$
		\item $<x>a$ -- initial state of DFA is x
	\end{itemize}
	
	
\end{frame}

\begin{frame}[fragile]
	
	\frametitle{User actions}	
	
	\begin{itemize}
		\item Default = input $\rightarrow$ output;
		\item yytext = matched token, yyleng = len(yytext);
		\item ECHO = printf(''\%s'', yytext);
		\item input(), unput(), output() macros;
		\item yymore(), yyless(yyleng - 1);
		\item yywrap() after EOF, default retval == 1;
		\item REJECT; means <<go to the next alternative>>;
		\item \#define YY\_USER\_ACTION.
	\end{itemize}
	
\end{frame}

\begin{frame}[fragile]
	
	\frametitle{Definitions}	
	
	Copied directly to lex.yy.c:
	\begin{itemize}
		\item Any line that is not a regex-action pair and begins with a whitespace;
		\item Anything between \{\% and \%\};
		\item Anything after \%\%.
	\end{itemize}
	
\end{frame}

\begin{frame}[fragile]
	
	\frametitle{Lex + YACC}	
	
	YACC uses retvals from yylex().
	
	So, each user action block should end with
	
	\begin{lstlisting}
return(token);
	\end{lstlisting}
	
	Link it with -lfl key.
	
\end{frame}

\begin{frame}[fragile]
	
	\frametitle{Start conditions}	
	
	\begin{lstlisting}
%START AA BB CC
%%
^a                {ECHO; BEGIN AA;}
^b                {ECHO; BEGIN BB;}
^c                {ECHO; BEGIN CC;}
\n                {ECHO; BEGIN 0;}
<AA>magic         printf("first");
<BB>magic         printf("second");
<CC>magic         printf("third");
	\end{lstlisting}

\end{frame}

\begin{frame}[fragile]
	
	\frametitle{EOF rules}

\begin{small}
	\begin{lstlisting}
%x quote
%%

...other rules for dealing with quotes...

<quote><<EOF>>   {
	error( "unterminated quote" );
	yyterminate();
}
<<EOF>>  {
	if ( *++filelist )
		yyin = fopen( *filelist, "r" );
	else
		yyterminate();
}
	\end{lstlisting}
\end{small}

\end{frame}
	
\begin{frame}[fragile]
	
	\frametitle{C++ scanners}
	
	\begin{itemize}
		\item \%option c++, or 'flex -+', or 'flex++'
		\item `lex.yy.c' $\rightarrow$ `lex.yy.cc', includes 'FlexLexer.h'
		\item FlexLexer \begin{itemize}
			\item const char* YYText() $\leftrightarrow$ yytext
			\item int YYLeng() $\leftrightarrow$ yyleng
			\item int lineno() const $\leftrightarrow$ \%option yylineno
		\end{itemize}
		\item yyFlexLexer : FlexLexer \begin{itemize}
			\item yyFlexLexer( istream* yyin, ostream* yyout )
			\item virtual int yylex()
		\end{itemize}
	\end{itemize}
	
\end{frame}

\begin{frame}[fragile]
	
	\frametitle{Notes}	
	
	\begin{itemize}
		\item Remember that . does not include newline;
		\item Don't use $(.|\backslash n)+$, it causes buffer overflow;
		\item REJECT; 
	\end{itemize}
	
\end{frame}

	%%%%%%%%%%%%%%%%%%%%%%%%%%%%%%
	
	\begin{frame}
		\center{\huge
			Dura lex, sed lex.
		}
	\end{frame}
	
\end{document} 