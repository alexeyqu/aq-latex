\documentclass[14pt]{beamer}
\usepackage[T2A]{fontenc}
\usepackage[utf8]{inputenc}
\usepackage[english,russian]{babel}
\usepackage{amssymb,amsfonts,amsmath,mathtext}
\usepackage{cite,enumerate,float,indentfirst}
\usepackage{booktabs}
\usepackage{listings}

\graphicspath{{images/}}

\usetheme{Pittsburgh}
\usecolortheme{whale}

\setbeamercolor{footline}{fg=blue}
\setbeamertemplate{footline}{
	\leavevmode%
	\hbox{%
		\begin{beamercolorbox}[wd=.333333\paperwidth,ht=2.25ex,dp=1ex,center]{}%
			Kulikov Alexey, ABBYY
		\end{beamercolorbox}%
		\begin{beamercolorbox}[wd=.333333\paperwidth,ht=2.25ex,dp=1ex,center]{}%
			Moscow, 2017
		\end{beamercolorbox}%
		\begin{beamercolorbox}[wd=.333333\paperwidth,ht=2.25ex,dp=1ex,right]{}%
			Page \insertframenumber{} of \inserttotalframenumber \hspace*{2ex}
	\end{beamercolorbox}}%
	\vskip0pt%
}

\newcommand{\itemi}{\item[\checkmark]}

\title{GNU Bison}
\subtitle{\footnotesize{}}
\author{\small{%
		~Kulikov Аlexey}\\%
	\vspace{30pt}%
	ABBYY%
	\vspace{20pt}%
}
\date{\small{Moscow, 2017}}

\begin{document}
	
	\maketitle

\begin{frame}[fragile]
	
	\frametitle{Flex EOF rules}

\begin{small}
	\begin{lstlisting}
%x quote
%%

...other rules for dealing with quotes...

<quote><<EOF>>   {
	error( "unterminated quote" );
	yyterminate();
}
<<EOF>>  {
	if ( *++filelist )
		yyin = fopen( *filelist, "r" );
	else
		yyterminate();
}
	\end{lstlisting}
\end{small}

\end{frame}
	
\begin{frame}[fragile]
	
	\frametitle{Flex C++ scanners}
	
	\begin{itemize}
		\item \%option c++, or 'flex -+', or 'flex++'
		\item `lex.yy.c' $\rightarrow$ `lex.yy.cc', includes 'FlexLexer.h'
		\item FlexLexer \begin{itemize}
			\item const char* YYText() $\leftrightarrow$ yytext
			\item int YYLeng() $\leftrightarrow$ yyleng
			\item int lineno() const $\leftrightarrow$ \%option yylineno
		\end{itemize}
		\item yyFlexLexer : FlexLexer \begin{itemize}
			\item yyFlexLexer( istream* yyin, ostream* yyout )
			\item virtual int yylex()
		\end{itemize}
	\end{itemize}
	
\end{frame}

\begin{frame}[fragile]
	
	\frametitle{What is Bison?}	
	
	\begin{itemize}
		\item CF grammar $\rightarrow$ LR parser with LALR(1) tables;
		\item Upward-compatible with Yacc;
		\item Is a part og the GNU Project;
		\item \href{https://www.gnu.org/software/bison/}{www.gnu.org/software/bison/};
		\item Used by: Ruby, PHP, GCC (before 2004), Go, Bash, MySQL, LilyPond, ...
	\end{itemize}
	
\end{frame}

\begin{frame}[fragile]
	\frametitle{Bison pipeline}
	\begin{itemize}
		\item[1] $ \text{*.y} \rightarrow \boxed{\text{ Bison }} \rightarrow \text{yyparse: *.tab.c, *.tab.h} $
		\item[2] $ \text{*.lex} \rightarrow \boxed{\text{ Lex }} \rightarrow \text{yylex: *.yy.c} $
		\item[3] $ \text{Input} \rightarrow \boxed{\text{ yylex }} \rightarrow \text{Token\_ID} $
		\item[4] $ \text{Token\_ID} \rightarrow \boxed{\text{ yyparse }} \rightarrow \text{Semantic Operation} $
	\end{itemize}
	
\end{frame}

\begin{frame}[fragile]
	
	\frametitle{Bison Grammar File}	
	
\begin{lstlisting}
	%{
	Prologue
	%}
	
	Bison declarations
	
	%%
	Grammar rules
	%%
	
	Epilogue
\end{lstlisting}

\end{frame}

\begin{frame}[fragile]
	
	\frametitle{Prologue}	
	
\begin{itemize}
	\item Contains definitions of: macros, functions, etc;
	\item Inserted into .c file before definition of yyparse;
	\item \textbf{\%code} inserts: \begin{itemize}
		\item (unqualified): into default location, \textit{verbatim};
		\item \textbf{requires}: into .h and .c before YYL/STYPE defs \\ good for writing YYL/STYPE overrides;
		\item \textbf{provides}: into .h and .c after YY*, token defs \\ good for other defs and decls;
		\item \textbf{top}: inserted near the top of .c file;
		\item \textbf{imports}: Java-specific, for import directives.
	\end{itemize}
\end{itemize}
	
\end{frame}


\begin{frame}[fragile]
	
	\frametitle{Notes}	
	
	\begin{itemize}
		\item YYLTYPE: Macro for the data type of yylloc;
		\item YYSTYPE: Macro for the data type of semantic values; int by default.
	\end{itemize}
	
\end{frame}

	\begin{frame}
	\center{\huge
		À bientôt.
	}
\end{frame}

\begin{frame}[fragile]
	
	\frametitle{Bison declarations}	
	
	\begin{itemize}
		\item All token type names must be declared;
		\item The first rule specifies the start symbol;
		\item \%token name $\rightarrow$ \#define
	\end{itemize}
	
\end{frame}

	%%%%%%%%%%%%%%%%%%%%%%%%%%%%%%
	

\end{document} 